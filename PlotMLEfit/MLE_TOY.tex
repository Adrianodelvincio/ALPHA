\documentclass[9pt]{beamer}
\usetheme{Boadilla}
\usepackage[utf8]{inputenc}
\usepackage{amsmath}
\usepackage{amsfonts}
\usepackage{amssymb}
\usepackage{graphicx}
\usepackage{subfig}
\usepackage[export]{adjustbox}
\author{Adriano Del Vincio, Germano Bonomi,\\ Simone Stracka}
\title[Alpha 2]{Studying Annihilation Distributions \hrule}
%\setbeamercovered{transparent} 
%\setbeamertemplate{navigation symbols}{} 
\logo{\includegraphics[width = 0.125\textwidth]{../logo/ALPHA_Logo.jpg}}
\newcommand{\nologo}{\setbeamertemplate{logo}{}}
\institute{University of Brescia, INFN Pisa} 
\titlegraphic{
\begin{figure}
\hspace{1.cm}
\includegraphics[width = 0.12\textwidth ]{../logo/logounibs.png}
%\hspace{1cm}
\includegraphics[width = 0.25\textwidth ]{../logo/1_Pisa_LOGO_SIGLA.pdf}
\end{figure}
}
\date{07/11/2023} 
\subject{ALPHA 2 spectroscopy, identifying the distinct contribution of $\overline{H}$ annihilation on the walls and the annihilation resulting from the residual gas.} 
%\setlength{\textfloatsep}{6 pt plus 1.0pt minus 2.0pt}
\begin{document}

\begin{frame}
\titlepage
\end{frame}

%\begin{frame}
%\tableofcontents
%\end{frame}

\begin{frame}{}

I tried to fit the distributions the radial distribution of the anti-hydrogen annihilation with analytic models. This should improve the results, avoiding statistical fluctuations:
The models are listed below:
\begin{itemize}
\item Pdf Mixing: the Normal distribution.
\item Pdf Uwlosses: Rayleigh distribution $\frac{r}{\sigma^{2}} e^{-\frac{r^{2}}{2 \sigma^{2}}}$.
\item Pdf Cosmic: $k \cdot x$.
\end{itemize}

The factor $k$ is the normalization constant. The Mixing, UW losses and cosmic data are fitted and the result are shown in the following slides.
\end{frame}

\begin{frame}{MIXING}

Mixing dataset represents almost pure data of anti-hydrogen annihilation on the walls. The radius distribution is fitted with a Gaussian.

\begin{figure}
\vspace{-2pt}
\includegraphics[width = 0.6\textwidth]{./SingleModel/GaussianFitMixing.pdf}
\end{figure}

\end{frame}

\begin{frame}{UWlosses}

The microwave losses represent an almost pure anti-hydrogen sample of annihilation events due to residual gas inside the trap.

\begin{figure}
\includegraphics[width = 0.75\textwidth]{./SingleModel/FitToUW.pdf}
\end{figure}

\end{frame}

\begin{frame}{Cosmic}

The cosmic distribution is obtained from dataset without anti-hydogren. 

\begin{figure}
\includegraphics[width = 0.7\textwidth]{./SingleModel/Cosmici_fit.pdf}
\end{figure}

\end{frame}

\begin{frame}{Pdfs normalized plotted together}
\begin{figure}
\includegraphics[width = 0.80\textwidth ,left]{PdfTogether.pdf}
\end{figure}
\end{frame}

\begin{frame}{Radial Density}

The histogram in $r$ variable doesn't account for the different area of the bin which is $2 \pi r \cdot dr$. So it is useful to divide per $2 \pi r$ to obtain the radial density of the events.

\begin{figure}[hbtp]
\centering
\includegraphics[width = 0.7\textwidth , left]{RadialDensity.pdf}
\caption{Radial density Pdf for f4 dataset}
\end{figure}
\end{frame}

\begin{frame}[t]{Toy simulation.}

To study the accuracy of the algorithm to reconstruct the various parameter, we have developed a "toy" simulation tool. The model to generate the data is:

\begin{equation}
Pdf_{total} = a \cdot Gauss_{mix} + b \cdot Rayleigh + c \cdot linearModel_{cosmic} 
\end{equation}

where $a,b,c$ represent the "weights" of the various contributions to the Pdf used to generate the data. The number of annihilation is indicated as $N_{sample}$. Once the data are generated, they are fitted with the model:

\begin{equation}
Pdf_{fit} = Nfit_{mix} \cdot Gauss_{mix} + Nfit_{uw} \cdot Rayleigh + Nfit_{bk} \cdot linearModel_{cosmic}
\end{equation}

The parameters of the fit are $Nfit_{mix}, Nfit_{uw}$ and $Nfit_{bk}$. The "true value" are defined as:

\begin{itemize}
\centering
\item $Ngen_{mix} = a \cdot N_{sample}$
\item $Ngen_{uw}  = b \cdot N_{sample}$
\item $Ngen_{bk}  = c \cdot N_{sample}$
\end{itemize} 
\end{frame}

\begin{frame}{Example of fit, Toy: $N_{sample} = 1000$, a = $33\%$, b = $33\%$, c = $33\%$}
\includegraphics[width = 0.65\textwidth , left]{N1000/FitToy(33,33,33).pdf} 
\end{frame}

\begin{frame}[t]{Toy: $N_{sample} = 1000$, a = $ 33\%$, b = $33\%$, c = $33\%$}

In this plot we have fixed the weight of each distribution to $33\%$, with $N_{sample} = 1000$ and $N_{trials} = 1000$. This first plot is made to ensure that the algorithm is able to reconstruct the parameters, and check the presence of a bias.
The variable of the histograms are: $\dfrac{Nfit - Ngen}{\sigma_{fit}}$

\begin{figure}
\includegraphics[width = 0.5\textwidth]{N1000/ToyNmix(33,33,33).pdf} 
\end{figure}

The distribution are normal, the fit is behaving as expected.

\end{frame}

\begin{frame}{weight variation $a$ for mix.}

Now we study how or if the coefficients of the fit $Nfit_{mix}, Nfit_{uw}$ and $Nfit_{bk}$ vary with the increment of the weight $a$. 
At fixed $c = 10\%$, $a$ is raised from $10\%$ to $80\%$ and $b$ is decreased accordingly. 

\begin{figure}
\vspace{-7pt}
\includegraphics[width = 0.9\textwidth , valign = t]{N1000/Nmix_and_Sigma(90,10,10).pdf}
\end{figure}
The number events is always $ N_{sample} = 1000$. For each value of the weight $a$  we\\ iterate 100 times ($N_{trials} = 100$) to study the reconstructed coefficients with the \\ variation of the weights.
\end{frame}

{\nologo
\begin{frame}{$N = 165$.}
Now we have done the same plot with $N = 165$, the same amount of data in \texttt{r68465\_uw\_exp\_freq4.vertex.csv} after applying \textit{cut1}. The value of $c$ is fixed to reproduce the number of expected events from background ($c = 6\%$).
\begin{figure}
\subfloat{\includegraphics[width = 0.9\textwidth]{N165/sigmaNmix(84,9,6).pdf}}
%\subfloat{\includegraphics[width = 0.49\textwidth]{N165/sigmaNmix(84,9,6).pdf}}
\end{figure}
\end{frame}
}
\begin{frame}{Fit to data.}

Pdf = Gaussian (Mixing) + Rayleigh (Uwlosses) + linear model (cosmic fixed).

Data taken from: \texttt{r68465\_uw\_exp\_freq4.vertex.csv}

\begin{figure}[hbtp]
\centering
\includegraphics[width = 0.70 \textwidth]{./SingleModel/TuttoAnalitico.pdf}
\end{figure}
\end{frame}

\begin{frame}[noframenumbering]
\begin{center}
\color{red}{\Large \bf{ADDITIONAL MATERIAL}}
\end{center}
\end{frame}

{\nologo
\begin{frame}{$N_{mix} - N_{fit}$ for $a = 46\%$, $b = 46\%$, $c = 6\%$.}
\begin{figure}
\includegraphics[width = 1\textwidth]{N165/ToyNmix(46,46,6).pdf}
\end{figure}
\end{frame}
}

{
\begin{frame}{$N_{uw}$ parameter of the fit model fixed}
The Toy simulation is tested fixing the $N_{uw}$ parameter of the fit model fixed. In the following plot the weight are $a = 46\%$, $b = 46\%$, $c = 6\%$, where $c$ is fixed in such a way to reproduce the number of expected background events in \texttt{r68465\_uw\_exp\_freq4.vertex.csv}, which correspond to $6\%$ of the total events.
\begin{figure}
\includegraphics[width = 0.8\textwidth]{N165/ToyNmixNbk,NuwFixed(46,46,6).pdf}
\end{figure}
\end{frame}
}

{\nologo
\begin{frame}{$N_{uw}$ parameter of the fit model fixed}

We study the bias $N_{mix} - N_{reconstructed}$ with the parameter $N_{uw}$ of the fit model fixed at $0$. For small value of $w_{mix}$, corresponding to small contribution of \textit{Mixing}  (and, conversely, a significant contribution of \textit{Uwlosses} pdf) we observe a large bias.

\begin{figure}
\subfloat{\includegraphics[width = 0.49\textwidth]{N165/Nmix,NuwFixed(84,9,6).pdf}}
\subfloat{\includegraphics[width = 0.49\textwidth]{N165/sigmaNmix,NuwFixed(84,9,6).pdf}}
\end{figure}
\end{frame}
}

\end{document}